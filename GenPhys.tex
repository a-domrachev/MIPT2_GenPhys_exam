\documentclass{letnab}
\pagestyle{fancy}
\usepackage{tabu}
\usepackage{units}
\usepackage{lscape}
\usepackage{ctable} % for \specialrule command
\usepackage{ulem}
\hyphenpenalty=1000
\usepackage{booktabs}
\rhead{Общая физика МФТИ}
\lhead{Экзамен: термодинамика} 

\renewcommand{\headrulewidth}{0.5pt}
\newcommand{\RomanNumeralCaps}[1]
{\MakeUppercase{\romannumeral #1}}

\newcommand{\chpr}[3]{\left(\dfrac{\partial #1}{\partial #2}\right)_#3}
\newcommand{\Chpr}[3]{\left(\frac{\partial #1}{\partial #2}\right)_#3}
\newcommand{\vtchpr}[4]{\dfrac{\partial}{\partial #1}\left(\dfrac{\partial #2}{\partial #3}\right)_#4}

\usepackage{xcolor}
\usepackage{hyperref}
\usepackage{upgreek}
\usepackage{wrapfig}
\usepackage{lipsum}


% Цвета для гиперссылок
\definecolor{linkcolor}{HTML}{000000} % цвет ссылок
\definecolor{urlcolor}{HTML}{799B03} % цвет гиперссылок

\hypersetup{pdfstartview=FitH,  linkcolor=linkcolor,urlcolor=urlcolor, colorlinks=true}
\setcounter{tocdepth}{1}

\begin{document}


	\begin{titlepage}
		

		
		
		\center % Center everything on the page
		
		
		
		%----------------------------------------------------------------------------------------
		%	HEADING SECTIONS
		%----------------------------------------------------------------------------------------
		
		\textsc{\LARGE Московский Физико-Технический Институт}\\[1,5cm] % Name of your university/college
		% Major heading such as course name
		\textsc{\Large Кафедра общей физики}\\[0.5cm]
		\textsc{\large Экзамен: термодинамика}\\[0.5cm] % Minor heading such as course title
		
		%----------------------------------------------------------------------------------------
		%	TITLE SECTION
		%----------------------------------------------------------------------------------------
		

		
		
		
		%----------------------------------------------------------------------------------------
		%	AUTHOR SECTION
		%----------------------------------------------------------------------------------------
		
			\begin{center} \large
				\emph{Авторство и верстка:} \href{https://vk.com/domrachev_alexey}{Алексей \textsf{Домрачев}}\\
				Благодарю за неоценимую помощь: \href{https://vk.com/shevtsovalexey}{Алексея \textsf{Шевцова}}, \href{https://vk.com/uteshevia}{Ивана \textsf{Утешева}} и \href{https://vk.com/detinin_roman}{Романа \textsf{Детинина}}
			\end{center}

		
		
		\begin{bottompar}
			\includegraphics[width = 80 mm]{logo.png}	\\[1,0cm]
			{\large \today}
		\end{bottompar}
		\vfill % Fill the rest of the page with whitespace
		
	\end{titlepage}



\tableofcontents
\newpage

 \section{\normalsize Термодинамическая система. Микроскопические и макроскопические параметры. Уравнение состояния (термическое и калорическое). Стационарные, равновесные и неравновесные состояния и процессы. }
\paragraph{Термодинамическая система.}\textbf{Cистема} --- совокупность рассматриваемых тел(\textit{частиц}), которые могут взаимодействовать между собой и с другими телами(\textit{внешняя среда}) посредством обмена веществом и энергией.\\
В термодинамике рассматриваются большие системы, называемые \textbf{термодинамическими системами}.
\paragraph{Микроскопические и макроскопические параметры.}\textbf{Микроскопическое состояние} --- состояние системы, определяемое заданием координат и импульсов (\textit{микропараметры}) всех составляющих систему частиц.\\
\textbf{Макроскопическое состояние} --- состояние системы, характеризующееся небольшим числом макропараметров($P$, $V$, $T$, $\rho$, $\eta$ и т.д.). Макропараметры подразделяются на внутренние и внешние.
\paragraph{Уравнение состояния (термическое и калорическое).}\textbf{Уравнение состояния} --- соотношение, связывающее параметры, описывающие состояние термодинамического равновесия.\\
\textbf{Термодинамическое равновесие} --- состояние, в котором прекращаются все макроскопические процессы: выравниваются давление и температура по объему системы, а скорости прямых и обратных реакций сравниваются.\\
\textbf{Калорическое} уравнение состояния --- зависимость типа $U = U(V,T)$. Пример такого уравнения $PV = (\gamma - 1)U$, где $\gamma$ --- показатель адиабаты, а $U$ - внутренняя энергия всех молекул\\
\textbf{Термическое} уравнение состояния --- зависимость типа $f(P,V,T)=0$\\
\paragraph{Стационарные, равновесные и неравновесные состояния и процессы.}
\textbf{Стационарным состоянием системы} называется состояние, в котором определяющие его параметры не меняются со временем [\textit{в замкнутой системе термодинамическое равновесие это стац. состояние}].\\
В \textbf{равновесном процессе} система непрерывно проходит (бесконечно близкие) равновесные состояния. Все прочие процессы являются неравновесными.\\
\textbf{Равновесным состоянием} является состояние системы, при которым компоненты системы находятся в ТДР, и, как следствие, неизменны их макроскопические параметры. \\
\textbf{Неравновесный процесс} --- процесс, на траектории (\textit{совокупность всех промежуточных состояний}) которого встречаются неравновесные состояния.\\
\textbf{Неравновесное состояние} --- параметры системы меняются от точки к точке с течением времени.\\
\textbf{Обратимым} называют процесс, который может протекать как в прямом, там и в обратном направлении, причем возможно возвращение системы и ее окружения в исходное (макроскопическое) состояние. Если это неосуществимо, то процесс \textbf{необратим}. Неравновесные процессы необратимы. \\
\textbf{Круговой процесс} --- замкнутый равновесный процесс.

 
 \section{\normalsize Работа, внутренняя энергия, теплота. Первое начало термодинамики.}

\paragraph{Работа.} Бесконечно малая \textbf{элементарная работа} $\delta A$, совершаемая газом при бесконечно малом квазистатическом расширении, в котором его объем увеличивается на $dV$ рассматриваемая в модели газа под поршнем. $F = PS$ ($P$ - const, т.к. перемещение малое) $\then$ при перемещении поршня на $dx:\,\delta A = F dx= PSdx=PdV $. Следует еще заметить , что в квазистатических процессах $\delta A = - \delta A_\text{внешн.}$.\\
\textbf{Работа конечного процесса}: $A = \int_{1 \rightarrow 2} \delta A $, она не является функцией состояния, т.к. зависит от пути перехода от 1 к 2.\\
\textbf{Адиабатическая оболочка} характерна тем, что при любых изменениях температуры окружающих тел состояние системы внутри оболочки неизменно. Значение всех прочих внешних параметров неизменны, например не совершается механическая работа. Изменить состояние системы можно путем механической работы.
\paragraph{Внутренняя энергия.} \textbf{Внутренней энергией $U$ системы} называется функция состояния, приращение которой во всяком процессе, совершенном системой в адиабатической оболочке, равно работе внешних сил над системой при переходе ее из начального равновесного состояния в конечное, также равновесное, т.е. $U_2 - U_1 = A_\text{внешн.} \rightarrow U$ --- функция состояния.
\paragraph{Теплота.} Пусть система заключена в жесткую теплопроводную оболочку $\then$ имеем чисто тепловой контакт системы  с внешней средой без совершения макроскопической работы, происходит \textbf{теплообмен}, сопровождающийся обменом внутренними энергиями соприкасающихся тел, т.е. \textbf{количество теплоты} --- приращение внутренней энергии в процессе чистого теплообмена. $Q = U_2 - U _1$ --- полученное тепло (не функция состояния!)
\paragraph{Первое начало термодинамики.} Теплота Q, полученная системой, идет на приращение ее внутренней энергии $\Delta U = U_2 - U_1$, и совершение системой работы
$$ \int_{1 \rightarrow 2}\delta Q = \int_{1 \rightarrow 2}dU + \int_{1 \rightarrow 2}\delta A \then Q = \Delta U + A$$
Если процесс круговой, то $U_1 = U_2$ и $Q=0$, то $A = 0 \then$ невозможен процесс, единственным результатом которой является производство работы без каких-либо изменений в других телах.

 
 \section{\normalsize Идеальный газ. Связь давления и температуры идеального газа с кинетической энергией его молекул. Уравнение состояния идеального газа.}
\paragraph{Идеальный газ.} \textbf{Идеальный газ} --- газ, расстояние между молекулами которого настолько велико, что их взаимодействием можно пренебречь, а его внутренняя энергия --- кинетическая энергия частиц.
\paragraph{Связь давления и температуры идеального газа с кинетической энергией его молекул.} Число молекул со скоростью $v$ в единице объема --- $n(v)$ и импульс одной молекулы $p_x=mv_x$. Тогда импульс переданный стенке молекулой --- $2p_x$. Число молекул, которые долетают до стенки за $dt: \frac{1}{2}n(v)Sv_xdt \then$ суммарное изменение импульса $\Delta p = p_x nSv_xdt \then$ полный импульс по всему группам молекул: $$\sum_{v} p_xnSv_xdt = F_x dt \then P = \dfrac{F_x}{S}= \sum_{v}p_xnv_x=n\overline{v_x p_x}=2n\dfrac{\overline{mv_x^2}}{2},\; n = \sum_v n(v)$$
Вследствие изотропии газа $\overline{v_xp_x} =\overline{v_yp_y} =\overline{v_zp_z} =\frac{1}{3} \overline{vp} \then P = \dfrac{2}{3}n\dfrac{\overline{mv^2}}{2}$,  так как $E_\text{кин.} =\\= 3/2kT \then P=nkT$
\paragraph{Уравнение состояния идеального газа.} \textbf{Уравнение состояния вещества} --- соотношение, связывающее параметры, описывающие состояния термодинамического равновесия вещества.\\
\textbf{Для идеального газа} $PV = \nu RT$, где $\nu = \dfrac{m}{\mu}$, $R$ --- универсальная газовая постоянная.

 
 \section{\normalsize Работа идеального газа в равновесных изотермическом и изобарическом процессах. Внутренняя энергия идеального газа.}
\paragraph{Работа идеального газа в равновесных изотермическом и изобарическом процессах.} 
Работа $\nu = 1$ моль идеального \textbf{изотермическом} расширении. $PV = RT =\\ = const \then A = \int_{V_1}^{V_2}PdV=RT\int_{V_1}^{V_2}\dfrac{dV}{V}=RT \ln(\dfrac{V_2}{V_1})\then A_T=RT \ln(\dfrac{V_2}{V_1})$ \\
Работа $\nu = 1$ моль идеального газа при \textbf{изобарном} расширении. $$\delta A = PdS\cdot dn = P dV_\text{эл.} \then \delta A_P = \int_{V_\text{слоя}}PdV_\text{эл.} = P \int_{V_\text{слоя}} dV_\text{эл.} = PdV \then A_P=P\Delta V$$
\paragraph{Внутренняя энергия идеального газа.} Опыт Джоуля: идеальный газ с температурой $T_1$, давлением $P_1$ находится в части адиабатической оболочки объемом $V_1$, вторая часть оболочки откачана до вакуума. Перегородку между частями убирают, в следствии расширения газа его температура не изменилась, а изменилось лишь давление и объем. Рассмотрим это эмпирическое наблюдение с точки зрения теории. По первому началу термодинамики  $Q = \Delta U + A$, причему $Q=0$ , т.к. газ находится в адиабатической оболочке и $A=0$, т.к. газ расширялся в вакуум $\then \Delta U = 0 \then U_2(T,V_2)=U_1(T,V_1) \then$ функция $U$ зависит лишь от второго параметра и $\left(\dfrac{\partial U}{\partial V}\right)_T = 0$ для идеального газа. Внутренняя энергия идеального газа зависит лишь от температуры, поскольку она определяется лишь кинетической энергией молекул.\\
Для одноатомного газа: $U = N\left<\dfrac{mv^2}{2}\right>+ U_0$, пусть для удобства $U_0=0 \then U= \\= N \dfrac{3}{2}kT = \dfrac{3}{2}\nu N_A kT = \dfrac{3}{2}\nu RT$ \\
В общем случае: $U = \dfrac{i}{2}\nu RT$, где $i$ - степени свободы газа.


 \section{ \normalsize Теплоёмкость. Теплоёмкости $C_V$ и $C_P$. Теплоёмкости идеального газа при постоянном объёме и давлении, соотношение Майера.}
\paragraph{Теплоёмкость.}  \textbf{Теплоёмкостью тела} называется отношение бесконечно малого количества теплоты $\delta Q$, полученного телом, к соответственному приращению его температуры $dT$ $$C = \dfrac{\delta Q}{dT}$$
\paragraph{Теплоёмкости $C_V$ и $C_P$.} \textbf{Удельная теплоемкость $c$} --- теплоемкость в расчёте на единицу массы.\\
\textbf{Молярная теплоемкость $C_\mu$} --- теплоемкость в расчёте на 1 моль.
\paragraph{Теплоёмкости идеального газа при постоянном объёме и давлении.}
$$dU = \left(\dfrac{\partial U}{\partial T}\right)_V dT + \left(\dfrac{\partial U}{\partial V}\right)_T dV  \then C = \dfrac{dU + PdV}{dT} = \left(\dfrac{\partial U}{\partial T}\right)_V + \left[\left(\dfrac{\partial U}{\partial V}\right)_T dV+P\right]\dfrac{dV}{dT}$$
При постоянном \textbf{объёме}: $C_v = \left(\dfrac{\partial U}{\partial T}\right)_V$\\
При постоянном \textbf{давлении}: $C_P= \left(\dfrac{\partial U}{\partial T}\right)_V + \left[\left(\dfrac{\partial U}{\partial V}\right)_T dV+P\right]\left(\dfrac{dV}{dT}\right)_P$\\
\paragraph{Cоотношение Майера.} Для \textbf{идеального газа} $\left(\dfrac{\partial U}{\partial V}\right)_T = 0$
$$C_v = \left(\dfrac{\partial U}{\partial T}\right)_V, C_P = \left(\dfrac{\partial U}{\partial T}\right)_V + P\left(\dfrac{dV}{dT}\right)_P \then C_P-C_V = P\left(\dfrac{dV}{dT}\right)_P\left.\right|_{PV=RT}=P\cdot\dfrac{R}{P}=R$$
\textbf{Cоотношение Майера} --- $C_P-C_v = R$

 
 \section{\normalsize Адиабатический и политропический процессы. Уравнение адиабаты и политропы идеального газа.}
\paragraph{Адиабатический и политропический процессы.} \textbf{Адиабатическим} называется процесс, происходящий в теплоизолированной системе ($\partial Q = 0$)\\
\textbf{Политропическим} называется процесс, происходящий при постоянной теплоемкости ($C=const$)\\
Примеры политропических процессов:
\begin{enumerate}
	\item Адиабатический: $C=0,\,n=\gamma,\,PV^\gamma=const$
	\item Изобарический: $C=C_p,\,n=0,\,P=const$
	\item Изохорический: $C=C_v,\,n=\infty,\,V=const$
	\item Изотермический: $C=\infty,\,n=1,\,T=const$
\end{enumerate}
\paragraph{Уравнение адиабаты и политропы идеального газа.}
$$ \delta Q = C_VdT + PdV =0;\; T =\dfrac{PV}{R} \then dT = \dfrac{d(PV)}{R}=\dfrac{PdV+VdP}{R}=\dfrac{PdV+VdP}{C_P-C_V}\then $$
$$\then C_V \dfrac{PdV+VdP}{C_P-C_V}+PdV = 0 \Leftrightarrow C_VPdV+C_VVdP+C_PPdV-C_VPdV=0$$
Введем $\gamma=\dfrac{C_P}{C_V}$, тогда $\gamma PdV+VdP = 0 \Leftrightarrow \gamma d(\ln V)+d(\ln P) = 0$\\
У идеального газа $C_V = const,\,C_p=const\then\gamma=const\then d(\ln PV^\gamma)=0$\\
В итоге получаем $PV^\gamma=const$ --- уравнение \textbf{Пуассона.}\\

Теперь выведем \textbf{уравнение политропы}.
$$ \delta Q = CdT = C_VdT+PdV \Leftrightarrow (C-C_V)\dfrac{dT}{T}=R\dfrac{dV}{V} \Leftrightarrow$$
$$\Leftrightarrow (C-C_V)\ln T = R\ln V + const \Leftrightarrow \ln T - \ln \left(V^{\tfrac{R}{C-C_V}}\right) + const \Leftrightarrow TV^{-\tfrac{R}{C-C_V}}=const $$
Введем показатель политропы $n = \dfrac{C-C_P}{C-C_V}$, тогда $TV^{n-1}=const$ --- \textbf{уравнение политропы}.

 
 \section{\normalsize Скорость звука в газах. Скорость истечения газа из отверстия.}
\paragraph{Скорость звука в газах.} Колебания плотности, связанные с ними колебания температуры в звуковой волне происходят так быстро, что из-за малой теплопроводности воздуха \textbf{теплообмен не играет никакой роли}! Разности температур между сгущениями и разрежениями воздуха в звуковой волне не успевают выравниваться $\then$ его распространение можно считать \textbf{адиабатическим}\\
В уравнении адиабаты $P\sim\rho^\gamma\then \left(\dfrac{\partial P}{\partial \rho}\right)_\text{адиаб.}=\gamma \dfrac{P}{\rho}=\gamma\dfrac{RT}{\mu}\then$\\$\then c_\text{зв}=\sqrt{\left(\dfrac{\partial P}{\partial \rho}\right)_\text{адиаб.}}=\sqrt{\gamma\dfrac{RT}{\mu}}$.\\ Для воздуха $\gamma=1.4;\;\mu=28.8;$ при $T=273$ К $c_\text{зв}\simeq330$~м/с
\paragraph{Истечение газа из отверстия.} Исследуем адиабатическое ламинарное течение. Пусть изначально газ находился в сосуде при давлении $P_1$ и температуре $T_1$, после он истекает в среду с температурой $T_2$ и давлением $P_2$, известны все эти величины кроме $T_2$.\\
Уравнение Бернулли:
\begin{equation}
\label{bernuli} 
\dfrac{v_1^2}{2}+\dfrac{P_1}{\rho_1}+gh_1+u_1=\dfrac{v^2_2}{2}+\dfrac{P_2}{\rho_2}+gh_2+u_2,\,
\end{equation}  $gh=const$, т.к. не меняется существенно вдоль трубки тока.  Введем $H=I=U+PV$ --- энтальпия. $i = u+Pv_\text{уд.}$ --- удельная энтальпия. Подставим i в (\ref{bernuli}) $$\dfrac{v_1^2}{2}+i_1=\dfrac{v_2^2}{2}+i_2$$
Если сосуд большой, а отверстие мало, то можно принять, что скорость газа в сосуде $v_1=0 \then v_2=\sqrt{2(i_1-i_2)}$. \\
В случае идеального газа ($C_V=const$): $i=u+\dfrac{P}{\rho}=\dfrac{C_VT}{\mu}+\dfrac{RT}{\mu}=\dfrac{C_PT}{\mu}\then$\\
$\then v =$ \fbox{$\sqrt{\dfrac{2}{\mu}C_P(T_1-T_2)}$}\\
\textbf{Вычислительная формула}: $\dfrac{P_1^{\gamma-1}}{T_1^\gamma}=\dfrac{P_2^{\gamma -1}}{T_2^\gamma}\Leftrightarrow T_2=T_1\left(\dfrac{P_2}{P_1}\right)^{\tfrac{\gamma-1}{\gamma}}\\$
$$v = \text{\fbox{$\sqrt{\dfrac{2}{\mu}C_PT_1\left[1-\left(\dfrac{P_2}{P_1}\right)^{\tfrac{\gamma -1}{\gamma}}\right]}$}}$$\\
Максимальная скорость достигается при истечении в вакуум: $v_\text{вак.}=\sqrt{\dfrac{2}{\mu}C_PT}$ или \\
$v_\text{вак.}=\sqrt{\dfrac{2}{\mu}\dfrac{\gamma}{\gamma-1}RT}=\sqrt{\dfrac{2}{\gamma -1}}c_\text{зв.}$

 
 \section{\normalsize Цикл Карно. КПД машины Карно. Теоремы Карно.}
\paragraph{Цикл Карно.} \textbf{Тепловая машина} --- устройство, преобразующее теплоту в работу или обратно и действующее строго периодически.\\
\textbf{Машина Карно} --- тепловая машина, работающая между двумя резервуарами с $T_1$ и $T_2$, причем $T_2<T_1$, по обратимому циклу, состоящему из двух изотерм и двух адиабат (циклу Карно).
\paragraph{КПД машины Карно.} \textbf{КПД тепловой машины} --- отношение работы, произведенной машиной за один цикл, к теплоте, поглощенной в ходе рассматриваемого цикла.
$$\eta = \dfrac{A}{Q_\text{н}}=\dfrac{Q_1-Q_2}{Q_1}=1-\dfrac{Q_2}{Q_1}<1$$
Рассчитаем \textbf{КПД машины Карно.} Рабочее тело - идеальный газ.\\
$Q_{12}=\delta U_{12}+A_{12}=RT_1 \ln\left(\dfrac{V_2}{V_1}\right) ,\;\; Q_{34}'=-Q_{34}=-A_{34}=-RT_2\ln\left(\dfrac{V_4}{V_3}\right)$\\
$T_1V_2^{\gamma-1}=T_2V_3^{\gamma -1}\then\dfrac{T_1}{T_2}=\left(\dfrac{V_3}{V_2}\right)^{\gamma-1},\;\;T_2V_4^{\gamma-1}=T_1V_1^{\gamma -1}\then\dfrac{T_1}{T_2}=\left(\dfrac{V_4}{V_1}\right)^{\gamma-1} $\\
$\dfrac{V_3}{V_2} = \dfrac{V_4}{V_1} \then Q_{34}=RT_2 \ln \left(\dfrac{V_2}{V_1}\right),$ тогда $\eta = 1 -\dfrac{Q_{34}'}{Q_{12}}=1-\dfrac{T_2}{T_1}$
\paragraph{Теоремы Карно.} На эту тему существует отличная 
\href{https://mipt.ru/education/chair/physics/S_II/method/Carnot.pdf}{методичка В.С. Булыгина} (надпись кликабельна)
  
 \section{\normalsize Холодильная машина. Тепловой насос. Эффективность холодильной машины и теплового насоса, работающих по циклу Карно.}
\paragraph{Холодильная машина и ее эффективность.} \textbf{Холодильная машина} --- устройство, преобразующее работу в тепло, отбирая его у более холодного, и действующая строго периодически.\\
\textbf{КПД холодильный машины} --- отношение отобранного у холодильника тепла к совершенной над рабочим телом работы. $$\eta_\text{х}=\dfrac{Q_2}{A'}=\dfrac{Q_2}{Q_1-Q_2}=\dfrac{Q_2/Q_1}{1-\frac{Q_2}{Q_1}}=\dfrac{1-\eta}{\eta}=\dfrac{1}{\eta}-1,$$ где $\eta$ --- КПД машины Карно, работающей между теми же резервуарами.
\paragraph{Тепловой насос и ее эффективность.} \textbf{Тепловой насос} --- устройство, аналогичное холодильной машине, передающее тепло более нагретому телу.\\
\textbf{КПД теплового насоса} --- отношение отданного рабочим телом тепла к совершенной над ним работой $\eta_\text{тн}=\dfrac{Q'}{A'}=\dfrac{1}{1-\frac{Q_2}{Q_1}}=\dfrac{1}{\eta}$ 
 
 \section{\normalsize  Второе начало термодинамики. Неравенство и равенство Клаузиуса. Энтропия. Закон возрастания энтропии. Энтропия идеального газа.}
\paragraph{Второе начало термодинамики.} \textbf{Формулировка Клаузиуса:} невозможен круговой процесс, единственным результатом которого был бы переход тепла от более холодного тела к более нагретому.\\
\textbf{Формулировка Томсона:} невозможен круговой процесс, единственным результатом которого было бы производство работы за счет охлаждения теплового резервуара. Формулировки Клаузиуса и Томсона эквивалентны. \\
\textbf{Формулировка Планка:} невозможно построить периодически действующую машину, единственным результатом которой было бы поднятие груза за счет охлаждения теплового резервуара.
\paragraph{Неравенство Клаузиуса.} Имеем $n$ тепловых резервуаров $R_1,\ldots,R_n$ достаточно больших, чтобы в процесса теплообмена $T_1,\ldots,T_n\simeq const.$\\
Таким образом система A совершила круговой процесс, заимствовав $Q_1$ у $R_1,\ldots,Q_n$ у $R_n$, совершив $A=Q_1+\ldots+Q_n$\\
После совершения цикла возьмем $R_0$ с $T_0$, также достаточно большой и $n$ машин Карно $K_1,\ldots,K_n$, включив их как показано на рисунке. Синхронность их работы и количество не важно.\\
\begin{minipage}{55 mm}
	\begin{figure}[H]
		\includegraphics[width=50mm]{ris.png}
	\end{figure}
\end{minipage}
\begin{minipage}{115mm}
	Для i-ой машины за 1 цикл: $1+\dfrac{Q_i'}{Q_{oi}}=1-\dfrac{T_i}{T_o} \Leftrightarrow\dfrac{Q_{oi}}{T_o}+\dfrac{Q_i'}{T_i}=0$.\\
	Суммируя по i: $Q_o=\sum_iQ_{oi}=-T_o\sum_i\dfrac{Q_i'}{T_i}$\\
	$Q_o$ --- общее количество теплоты, отданное $R_o$. Объединим все $n$ циклов машин Карно с циклом A в один большой: $R_o$ отдал $Q_o$; $R_1$ отдал $Q_1+Q_1';\ldots\;R_n$ отдал $Q_n+Q_n'$. Совершена работа $A=Q_o+(Q_1+Q_1')+\ldots(Q_n+Q_n')$. В силу больших размеров $R_1,\ldots,R_n$ выберем в согласии с постулатом Томсона--Планка $Q_1',\ldots,Q_n'$ так, что \\
	$Q_i'+Q_i=0,\,i=\overline{1,n}\then$ все тепловые резервуары вернутся в исходное состояние, а $R_o$ отдаст $Q_o=T_o\sum_{i=1}^{n}\dfrac{Q_i}{T_i}$\\
\end{minipage}
Таким образом совершается круговой процесс, за который отдано $Q_o$ и совершена работа $A=Q_o$. Других изменений не произошло. Тогда $A\leqslant0$ из постулата Томсона--Планка, значит $Q_o\leqslant0$. Переходя к пределу бесконечно большого числа тепловых резервуаров $R_1,\ldots,R_n,\dots$, обменивающихся бесконечно малыми порциями тепла с A и $R_o$ получаем:
$$\oint\dfrac{\delta Q}{T}\leqslant0\text{ --- \textbf{неравенство Клаузиуса}}$$
$T$ --- температура теплового резервуара, с которым система в данным момент обменивается теплом. В квазистатическом цикле под $T$ можно понимать температуру окружающей среды, так как обе температуры одинаковы.\\
Квазистатический процесс обратим, следовательно справедливо $\oint\dfrac{\delta Q'}{T}\leqslant0$, где $\delta Q'$ --- элементарное количество теплоты, получаемое системой на отдельных участках процесса. Так как процесс идет через те же состояния, то $\delta Q=-\delta Q'\then\oint\dfrac{\delta Q}{T}\geqslant$. А такое соотношение верно только тогда, когда $$\oint_\text{квазист.}\dfrac{\delta Q}{T}=0\text{ --- \textbf{равенство Клаузиуса}}$$
\paragraph{Энтропия.} Рассмотрим 2 способа перехода из 1 в 2, каждый из которых --- квазистатический процесс. Объединим их в круговой 1\RomanNumeralCaps{1}2\RomanNumeralCaps{2}1 и применим равенство Клаузиуса.
$$\int_{1\RomanNumeralCaps{1}2}\dfrac{\delta Q}{T} + \int_{2\RomanNumeralCaps{2}1}\dfrac{\delta Q}{T}=0\Leftrightarrow\int_{1\RomanNumeralCaps{1}2}\dfrac{\delta Q}{T}=\int_{1\RomanNumeralCaps{2}2}\dfrac{\delta Q}{T}\text{ --- приведенное количество теплоты.}$$
Таким образом приведенное количество теплоты, полученное системой при любом квазистатическом круговом процессе равно нулю \textbf{или} приведенное количество теплоты, полученное системой в квазистатическом процессе, не зависит от пути перехода, а определяется лишь начальным и конечным состояниями.\\
Отсюда: \textbf{энтропия} --- функция состояния системы, определенная с точностью до константы.
$$\Delta S \equiv \int_{1\rightarrow2}\dfrac{\delta Q}{T},\;\;dS=\left(\dfrac{\delta Q}{T}\right)_\text{кваз.}$$
\paragraph{Энтропия идеального газа.}
\textbf{Для идеального газа:} $$\delta Q=C_VdT+PdV=C_V(T)dT+R\dfrac{dV}{V}T\Leftrightarrow dS=\dfrac{\delta Q}{T}=\nu C_V(T)\dfrac{dT}{T}+\nu R\dfrac{dV}{V}$$
$$S=\int C_V(T)\dfrac{dT}{T}+R\int \dfrac{dV}{V}$$
Если $C_V$ не зависит от $T$, то $S=\nu[C_v\ln(T)+R\ln\left(V\right)]+const$. Всякий адиабатический процесс с $\delta Q=0\then dS=0\then S=const$.\\
\paragraph{Закон возрастания энтропии.} Энтропия адиабатически изолированной системы не может убывать, она либо растет, либо постоянна. \\Система может переходить из 1 в 2 необратимо по \RomanNumeralCaps{1}. Вернем ее квазистатически по какому-либо \RomanNumeralCaps{2}. Тогда
$\oint\dfrac{\delta Q}{T}=\int_\RomanNumeralCaps{1}\dfrac{\delta Q}{T}+\int_{\RomanNumeralCaps{2}}\dfrac{\delta Q}{T}\leqslant0$.
Так как $\int_{\RomanNumeralCaps{2}}\dfrac{\delta Q}{T}=S_1-S_2\then S_2-S_1\geqslant\int_{1\rightarrow2}\dfrac{\delta Q}{T}$. Если система адиабатически изолирована, то $\delta Q=0\then S_2\geqslant S_1$.


  
 \section{\normalsize  Термодинамические потенциалы. Метод получения соотношений Максвелла (соотношений взаимности). }
\paragraph{Термодинамические потенциалы.} \textbf{Термодинамические потенциалы} --- функции определённых наборов термодинамических параметров, позволяющие находить все термодинамические характеристики системы, как функции этих параметров.
\paragraph{Метод получения соотношений Максвелла (соотношений взаимности).} У Лёши Шевцова он представлен на примере вывод одного из потенциалов. В Сивухине похожая ситуация, так что пока что опустим этот пункт, а в следующем билете вы увидите, что это за метод и в чем его суть.

   
 \section{\normalsize Свободная энергия Гельмгольца, термодинамический потенциал Гиббса. Уравнения Гиббса--Гельмгольца.}
\paragraph{Свободная энергия Гельмгольца.} $$\Psi=\Psi(T,V)=U-TS\then d\Psi=-PdV-SdT\then -S=\left(\dfrac{\partial \Psi}{\partial T}\right)_V,\;-P=\left(\dfrac{\partial\Psi}{\partial V}\right)_T\then$$ $$\then\dfrac{\partial^2 \Psi}{\partial V\partial T}=\dfrac{\partial^2 \Psi}{\partial T\partial V}\Leftrightarrow\dfrac{\partial}{\partial V}\left(\dfrac{\partial \Psi}{\partial T}\right)_V=\dfrac{\partial}{\partial T}\left(\dfrac{\partial \Psi}{\partial V}\right)_T\Leftrightarrow\underline{\left(\dfrac{\partial S}{\partial V}\right)_T=\left(\dfrac{\partial P}{\partial T}\right)_V}$$
Вот такой метод получения данных соотношений через двойное дифференцирование и называется \textbf{методом Максвелла}.
\paragraph{Термодинамический потенциал Гиббса.} $$\Phi=\Phi(T,P)=U-TS+PV\then d\Phi=VdP-SdT\then V =\chpr{\Phi}{P}{T},\;-S=\chpr{\Phi}{T}{P}\then$$
$$\then\vtchpr{T}{\Phi}{P}{T}=\vtchpr{P}{\Phi}{T}{P}\Leftrightarrow\underline{\chpr{V}{T}{P}=-\chpr{S}{P}{T}}$$
\paragraph{Уравнения Гиббса-Гельмгольца.} Из свободной энергии Гельмгольца $U = \Psi + TS$ подставим $S$ полученное при частном дифференцировании, тогда $U = \Psi -T\chpr{\Psi}{T}{V}$. Проведем аналогичные действия с термодинамическим потенциалом Гиббса: \\$I = \Phi - T\chpr{\Phi}{T}{P}$. Данные соотношения называются \textbf{Уравнениями Гиббса-Гельмгольца}

 
 \section{\normalsize  Внутренняя энергия как термодинамический потенциал. Связь производной $\Chpr{U}{V}{T}$с термическим уравнением состояния.}
\paragraph{Внутренняя энергия как термодинамический потенциал.} 
$$U = U(S,V) \then dU=TdS-PdV\then T=\chpr{U}{S}{V},\;-P=\chpr{U}{V}{S}\then$$
$$\then\vtchpr{V}{U}{S}{V}=\vtchpr{S}{U}{V}{S}\Leftrightarrow\underline{\chpr{T}{V}{S}=-\chpr{P}{S}{V}}$$
\paragraph{Связь производной $\Chpr{U}{V}{T}$с термическим уравнением состояния.}
$$dU=TdS-PdV\then\chpr{U}{V}{T}=T\chpr{S}{V}{T}-P=T\chpr{P}{T}{V}-P$$
Полученное соотношение устанавливает связь между калорическим и термическим уравнениями состояния.
   
 \section{\normalsize Разность $C_P-C_V$ для произвольной термодинамической системы.} 
$$CdT=dU+PdV=\chpr{U}{T}{V}dT+\left[P+\chpr{U}{V}{T}\right]dV$$
Полагая $C=C_P$:
$$C_P-C_V=\left[P+\chpr{U}{V}{T}\right]\chpr{V}{T}{P}$$
\text{воспользуемся соотношением из п. 13.2}
$$C_P-C_V=T\chpr{P}{T}{V}\chpr{V}{T}{P}=\left|\chpr{P}{T}{V}\chpr{T}{V}{P}\chpr{V}{P}{T}=-1\right|=-T\dfrac{\Chpr{V}{T}{P}^2}{\Chpr{V}{P}{T}}$$
Поскольку $\chpr{V}{P}{T}<0$, то $C_P-C_V>0$.
 
 \section{\normalsize Теплофизические свойства твердых тел. Адиабатическое растяжение упругого стержня.}
\paragraph{Теплофизические свойства твердых тел.} \textbf{Теплофизические свойства материала} -- свойства, характеризующие поведение этого материала при изменении температуры, как-либо: теплоемкость, теплопроводность, коэффициенты теплового расширения, температура плавления.
\paragraph{Адиабатическое растяжение упругого стержня.} \textbf{Уравнение состояния} --- $f~=~f(l,T)$. При $f=0$ $l(T,0)=l_0(1+\alpha(T-T_0))$, где $l_0=l(T_0,0)$, $\alpha$ --- коэффициент линейного температурного расширения. При $T=const$ согласно закону Гука: $$\dfrac{\Delta l}{l}=\dfrac{f}{ES_\perp}=\dfrac{l(T,f)-l(T,0)}{l(T,0)}\then f=ES_\perp\left(\dfrac{l}{l_0(1+\alpha[T-T_0])}-1\right)$$
В большинстве случаев тепловая деформация мала, $\alpha|T-T_0|\ll1\then$
$$\then f=ES_\perp\left(\dfrac{l}{l_0}(1-\alpha[T-T_0])-1\right)$$
Предположен, что стержень окружен адиабатической оболочкой. При квазистатической деформации $dS=\chpr{S}{T}{l}dT+\chpr{S}{l}{T}dl=0\Rightarrow dT=-\dfrac{\Chpr{S}{l}{T}}{\Chpr{S}{T}{l}}dl;\;\chpr{S}{T}{l}=\left(\dfrac{\delta Q}{\partial T}\right)_l~\dfrac{1}{T}~=~\dfrac{C_l}{T}$ \\
$$PdV=-\sigma d(s_\perp l)=-\sigma S_\perp dl=-fdl\Rightarrow dU=TdS+fdl$$
$$\Psi=U-TS,\,d\Psi=-SdT+fdl\Rightarrow\text{ метод Максвелла --- } \chpr{S}{l}{T}=\chpr{f}{T}{l}$$
Откуда 
$$dT=\dfrac{T}{C_l}\chpr{f}{T}{l}dl$$
Используя $f:\,\Delta T=\int_{l_0}^{l}\dfrac{T}{C_l}\chpr{f}{T}{l}dl=-\dfrac{ES_\perp \alpha}{2C_ll_0}T(l^2-l_0^2)\simeq\underline{-\dfrac{ES_\perp\alpha}{C_l}T(l-l_0)}$\\
При адиабатическом растяжении ($l>l_0$) температура стержня понижается из-за совершения работы против внутренних сил притяжения молекул. Для идеального стержня ($E=const$):
$$U_\text{деф.}=V\dfrac{E\varepsilon^2}{2}=\Psi_\text{деф.}$$
Внутренняя энергия деформации совпадает со свободной энергией и явно не зависит от температуры.

 
 \section{\normalsize Фазовые переходы первого рода. Уравнение Клапейрона--Клаузиуса. Фазовое равновесие «жидкость--пар». Критическая точка.}
\paragraph{Фазовые переходы первого рода.} \textbf{Фаза} --- макроскопическая физическая однородная часть вещества, отделенная от остальных частей системы границами раздела, так что она может быть извлечена из системы механическим путем.\\
\textbf{Фазовый переход} --- переход вещества из одной фазы в другую при изменении внешних условий (температуры, давления, полей) при подводе или отводе тепла и т.д. \\
Величины, пропорциональные объему подсистемы называются \textbf{экстенсивными} ($V,U,S$), а не зависящие от объема выделенной подсистемыы --- \textbf{интенсивными} ($T,P,\rho$).\\
Фазовые превращения, при которых первые производные удельного термодинамического потенциала $\varphi(T,P)$ меняются скочкообразно называются \textbf{фазовыми переходами первого рода}\\
$v=\chpr{\varphi}{P}{T},\,s=-\chpr{\varphi}{T}{P}\Rightarrow$ скачкообразно меняется плотность ($\rho\simeq\frac{1}{v})$. Отлична от нуля теплота фазового перехода $q_{12}=T(s_2-s_1)$. \\
Плавление, испарение, возгонка, кристаллизация сопровождаются выделением или поглощением тепла, поэтому относятся к Ф.П. \RomanNumeralCaps{1} рода.
\paragraph{Уравнение Клапейрона--Клаузиуса.} Условия равновесия системы: 
\begin{enumerate}[1)]
	\item $P=const$ --- условие механического равновесия;
	\item $T=const$ --- условие теплового равновесия;
	\item $\varphi=const$ --- условие фазового перехода.
\end{enumerate}
Обоснование 3) : \\
	Рассмотрим двух фазную сиситему в жесткой адиабатической оболочке. Проведем в системе некий бесконечно малый процесс, в холе которого $T=const,\,P=const$ в обоих подсистемах и равны между собой, тогда:\\ $dU_1=TdS_1-PdV_1+\varphi_1dN_1;\;$\\$dU_2=TdS_2-PdV_2+\varphi_2dN_2$;\\
	Полная энтропия системы $S=S_1+S_2$, а в следствие изолированности $dU_1=-dU_2\Rightarrow\\\Rightarrow TdS=(\varphi_1-\varphi_2)dN_2$.В состоянии термодинамического равновесия энтропия максимальная, значит $dS=0\Rightarrow\varphi_1=\varphi_2$\\
    $d\varphi_1=-s_1dT+v_1dP,\quad d\varphi_2=-s_2dT+v_2dP,\quad\varphi_1=\varphi_2\Rightarrow d\varphi_1=d\varphi_2\Rightarrow$\\
    $\Rightarrow(s_2-s_1)dT=(v_2-v_1)dP\Leftrightarrow\dfrac{dP}{dT}=\dfrac{s_2-s_1}{v_2-v_1},\quad q_{12}=T(s_2-s_1)$ --- теплота фаз перехода в расчете на 1 частицу $\then \dfrac{dP}{dT}=\dfrac{q_{12}}{T(ы_2-v_1)}$ --- \textbf{Уравнение Клапейрона--Клаузиуса}.
\paragraph{Фазовое равновесие «жидкость--пар».}$\;$\\
\begin{minipage}{75mm}
	\begin{figure}[H]
		\includegraphics[width=65mm]{Klap.png}
	\end{figure}
\end{minipage}
\begin{minipage}{100mm}
	 Примем $q_{12}=q=const\Rightarrow\dfrac{dP}{dT}=\dfrac{q}{T(v_1-v_2)}$\\
	 Так как $v_1\gg v_2$:\\
	 $\simeq\left.\dfrac{q}{Tv_1}\right|_{Pv_1=kT}=\dfrac{q}{kT^2}P\Leftrightarrow\int_{P_0}^{P}=\dfrac{dP}{P}=\dfrac{q}{k}\int_{T_0}^{T}\dfrac{dT}{T^2}\Rightarrow\\
	 \Rightarrow P=P_0exp\left(\dfrac{q}{kT_0}-\dfrac{q}{kT}\right), P(T_0)=P_0$\\
	 \textbf{Критическая точка(К)} --- точка, в которой обрывается кривая фазового равновесия и исчезает разница между фазами. При наличии К всегда есть путь, в каждый момент которого вещество однородно, т.е. не возникает граница раздела.
\end{minipage}
 I --- пересекая кривую фазового равновесия (с образованием двухфазовой системы)\\
II --- в обход К (сохраняя пространственную однородность)\\
Для воды: $T_\text{кр.}=647.3\text{ К, }P_\text{кр.}=22.12\text{ МПа}$
  
 \section{\normalsize Диаграмма фазового равновесия «твёрдое тело--жидкость--пар». Тройная точка.}
\paragraph{Диаграмма фазового равновесия «твёрдое тело--жидкость--пар».}$\;$\\
\begin{minipage}{75mm}
	\begin{figure}[H]
		\includegraphics[width=65mm]{ris17.png}
	\end{figure}
\end{minipage}
\begin{minipage}{100mm}
	Рассмотрим трех фазную систему. Для равновесия необходимо:
	\begin{enumerate}[(1)]
		\item $\varphi_1(P,T)=\varphi_2(P,T)$ --- кривая испарения 12
		\item $\varphi_2(P,T)=\varphi_3(P,T)$ --- кривая плавления 23
		\item $\varphi_1(P,T)=\varphi_3(P,T)$ --- кривая возгонки 31
	\end{enumerate}
	Все они пересекаются в т.А
\end{minipage}
\paragraph{Тройная точка.}
Точка А называется \textbf{тройной точкой}, три фазы мыгоут находится в равновесии друг с другом, лишь в этой точке, имеющей конкретный параметры. В малой окрестности тройной точки можно провести круговой изотермический процесс, для которого справедливо
$$\oint\dfrac{\delta Q}{T}=0\underset{T=const}{\longmapsto}\oint\delta Q \Rightarrow q_\text{13}=q_\text{32}+q_\text{21}$$
   
 \section{\normalsize Зависимость теплоты фазового перехода от температуры.}
$$q=T(S_2-S_1)\text{ --- удельная теплота фазового превращения.}$$
    
 \section{\normalsize Уравнение Ван-дер-Ваальса как модель неидеального газа. Изотермы газа Ван-дер-Ваальса. Критические параметры. Приведённое уравнение Ван-дер-Ваальса, закон соответственных состояний.}
\paragraph{Уравнение Ван-дер-Ваальса.}
$$\left(P+\dfrac{a\nu^2}{V^2}\right)\left(V-\nu b\right)=RT$$

\begin{wrapfigure}{R}{5cm}
	\includegraphics[width=65mm]{ris19.png}
	\caption{Учет конечности размеров молекул}
\end{wrapfigure}
Выведем его, основываясь на $PV=\nu RT$:\\
Во-первых, учтем размеры молекул: \\$\dfrac{4}{3}\pi(2r)^3=8\cdot\dfrac{4}{3}\pi r^3$ --- недоступный объем для второй частицы $\Rightarrow$ в расчете на 1 молекулу $$\dfrac{1}{2}(8\cdot\dfrac{4}{3}\pi r^3)=4V_0$$
где $V_0=4\pi r^3/3$ --- объем одной молекулы. 
В результате объем, разрешенный для движения молекул, cоставит
$$V_\text{доп.} =V-\nu b,\quad$$
$$ b\simeq4\cdot\text{(объем молекулы в одном моле)}=4N_AV_0$$
Во-вторых, учтем, что молекулы притягиваются друг к другу. Одним из механизмов такого притяжения может быть перераспределение зарядов и образование диполей (см. рис 2).

\begin{wrapfigure}{l}{4cm}
	\label{dipol}
	\includegraphics[width=40mm]{ris19_2.png}
	\caption{\small Молекулы--диполи притягиваются друг к другу}
\end{wrapfigure}
Давление газа определяется столкновениями молекул со стенкой. Сила, действующая на молекулу у стенки со стороны газа $\sim n$, где $n$ --- число частиц. Частота соударений $\sim n$, значит давление уменьшается на $\Delta P\sim n^2$. Переходя от плотности $n$ к объему $V$ по формуле $n=\dfrac{\nu N_A}{V}$, мы можем записать поправку к давлению в виде $\Delta P=a_1n^2=a(\nu/V)^2.$ Окончательно это даёт
$$\left(P+\dfrac{a\nu^2}{V^2}\right)\left(V-\nu b\right)=RT$$
Величины $a$ и $b$ называется параметрами Ван-дер-Ваальса. Параметре $a$ учитывает притяжение, а $b$ --- отталкивание молекул.
\paragraph{Изотермы газа Ван-дер-Ваальса. Критические параметры.}

\begin{wrapfigure}{r}{8cm}
	\label{VdV}
	\includegraphics[width=80mm]{ris19_3.png}
	\caption{Изотермы Ван-дер-Ваальса}
\end{wrapfigure}
Уравнение Ван-дер-Ваальса: $$PV^3-(RT+\\+Pb)V^2+aV-ab=0$$ имеет один или три корня. В случае 1 это изотерма MN, а трех --- три пересечения в точках A, C, E изотермы LABCDEG. При некоторой $T$ $V_1=V_2=V_3$. Такая температура называется \textbf{критической} (изотерма FKH). Для нахождения критических давления, температуры и объема воспользуемся уравнением: $P_\text{к.}~(V~-~V_\text{к.})^3~=~0$\\

\begin{equation*}
\begin{cases}
P_\text{к.}V_\text{к.}^3=ab\\
3P_\text{к.}V_\text{к.}^2=a\\
3P_\text{к.}V_\text{к.}=RT_\text{к.}+P_\text{к.}b
\end{cases}
\Leftrightarrow
\begin{cases}
P_\text{к.}=\dfrac{a}{27b^2}\\
V_\text{к.}=3b\\
T_\text{к.}=\dfrac{8a}{27Rb}
\end{cases}
\end{equation*}
\paragraph{Приведённое уравнение Ван-дер-Ваальса} $\varphi=\dfrac{V}{V_\text{к.}},\quad\pi=\dfrac{P}{P_\text{к.}},\quad\tau=\dfrac{T}{T_\text{к.}}$ --- приведённые параметры. Тогда $V~=~3b\varphi,\quad P~=~\dfrac{a\pi}{27b^2},\\T~=~\dfrac{8a}{27Rb}~\tau\Rightarrow$ \textbf{приведённое уравнение Ван-дер-Ваальса}:
\begin{equation*}
\left(\pi+\dfrac{3}{\varphi^2}\right)(\varphi-1/3)=8/3\tau
\end{equation*}
\paragraph{Закон соответственных состояний.}Уравнение Ван-дер-Ваальса содержит только 3 параметра: $a,\,b,\,R$. Всякое уравнение, обладающее этим свойством, записанное в переменных $\varphi,\,\pi,\,\tau$ должно быть также одинаковым для всех веществ. Это положение есть \textbf{закон соответственных состояний}.\\
\textbf{Соответственными} называют состояния разных веществ, которые имеют одинаковые $\varphi,\,\pi,\,\tau$. Следовательно: если для различных веществ из трех параметров $\varphi,\,\pi,\,\tau$ совпадают значения каких-либо двух, то совпадут и третьи.

    
 \section{\normalsize Метастабильные состояния: переохлажденный пар, перегретая жидкость. Устойчивость состояний. Правило Максвелла.}
\paragraph{Метастабильные состояния: переохлажденный пар, перегретая жидкость. Устойчивость состояний.} При специальных условия могут быть реализованы участки AG --- перенасыщенный пар и LB --- перегретая жидкость (см. рис. 3). Эти состояния называют \textbf{метастабильными}. Каждое существует, пока его менее устойчивая фаза не граничит с другой --- более устойчивой. Например, перенасыщенный пар переходит в насыщенный, если в него попадет капля воды.
\paragraph{Правило Максвелла.}Положение горизонтального участка определяется с помощью равенства Клазиуса $\oint\dfrac{\delta Q}{T}=0$. Из G в L можно перейти двумя путями: по GCL двухфазного состояния и по теоретической изотерме физически однородного вещества GACBL, содержащей неустойчивый участок ACB. Применим равенство Клаузиуса к квазистатическому круговому процессу GCLBCAG: $T=const\Rightarrow\oint\delta Q=0$, кроме того $\delta Q~=~dU~+~PV,\quad \\\oint dU=0\Rightarrow\oint PdV=0$ или $\int_{\text{GCL}}PdV+\int_{\text{LBCAG}}PdV=0$ или $\int_\text{LCG}PdV=\int_\text{LBCAG}PdV$\\
Значит площади QLGR и QLBCAGR равны, значит LG надо проводить так, чтобы $$S_\text{LBC}=S_\text{CAG}$$ Это и есть \textbf{правило Максвелла.}

     
 \section{\normalsize Внутренняя энергия и энтропия газа Ван-дер-Ваальса. Равновесное и неравновесное расширение газа Ван-дер-Ваальса в теплоизолированном сосуде.
}
\paragraph{Внутренняя энергия и энтропия газа Ван-дер-Ваальса.} Рассмотрим $U=U(T,V)$, тогда $dU=\chpr{U}{T}{V}dT+\chpr{U}{V}{T}dV=C_VdT+\left(T\chpr{P}{T}{V}-P\right)dV$. Для $\nu$ = 1 моль, предполагая $C_V=const,\ dU=C_VdT+\dfrac{a}{V^2}dV$
$$U=C_VT-\dfrac{a}{V}$$
С ростом объема и, следовательно, расстояния между молекулами (при $T=const$) внутренняя энергия газа растет.\\
Рассмотрим $S=S(T,V)$, тогда $dS=\chpr{S}{T}{V}dT+\chpr{S}{V}{T}dV=\dfrac{C_V}{T}dT+\chpr{P}{T}{V}dV$. Для $\nu=1$ моль, $C_V=const,\ dS=\dfrac{C_V}{T}dT+\dfrac{R}{V-b}dV$ 
$$S=S_0+C_V\ln\left(\dfrac{T}{T_0}\right)+R\ln\left(\dfrac{V-b}{V_0-b}\right)$$
\paragraph{Равновесное и неравновесное расширение газа Ван-дер-Ваальса в теплоизолированном сосуде.} Рассмотри \textbf{свободное расширение газа в вакуум.} Пусть в начальный момент газ Ван-дер-Ваальса находился в сосуде, занимая в нем объем $V$. После удаления перегородки газ получил возможность свободно расшириться до объема $V_2(V_2>V_1)$. Считая, что сосуд окружен теплоизолированной оболчкой найдем $\Delta T$ после установления равновесия. Поскольку $\delta Q=0$ и $\delta A=0$, то $dU=0$, а для газа Ван-дер-Ваальса\\ $U_{1,2}=C_VT_{1,2}-\dfrac{a}{V_{1,2}}\Rightarrow\Delta T=T_2-T_1=-\dfrac{a}{C_V}\left(\dfrac{1}{V_1}-\dfrac{1}{V_2}\right)<0$ значит газ в данном процессе \textbf{охалждается}.\\
При расширении газа работа совершается против сил притяжения молекул. Эта работа производится за счет кинетической энергии и, значит, сопровождается охлаждением газа.
      
 \section{\normalsize Эффект Джоуля--Томсона (дифференциальный и интегральный). Температура инверсии.} 
\paragraph{Дифференциальный эффект Джоуля--Томсона.} Рассмотрим процесс Джоуля--Томсона: адиабатическое стационарное течение газа через пористую перегородку, под действием разности давлений

\begin{wrapfigure}{L}{7cm}
	\includegraphics[width=70mm]{ris22_1.png}
\end{wrapfigure}
по разные стороны от пробки. Изменение температуры в этом процессе --- \textbf{эффект Джоуля-Томсона}. Течение медленное, следовательно можно пренебречь кинетической энергией. Тогда по первому началу термодинамики $Q=0\Rightarrow\\0=\Delta U+A=U_2-U_1+P_2V_2-P_1V_1=I_2-I_1\Leftrightarrow I_1=I_2$. В процессе Джоуля--Томсона $I=const$.\\
Пусть теперь по разные стороны от перегородки разность давлений мала. Найдем $\Delta T:$ $$\Delta I = \chpr{I}{T}{P}\Delta T+\chpr{I}{P}{T}\Delta P=0\Rightarrow\chpr{I}{T}{P}=C_P,\ \chpr{I}{P}{T}=V-T\chpr{V}{T}{P}\Rightarrow$$
$$\Rightarrow\left(\dfrac{\Delta T}{\Delta P}\right)_I=\dfrac{T\Chpr{V}{T}{P}-V}{C_P}$$
Если газ идеальный, то $V=\dfrac{RT}{P},\ T\chpr{V}{T}{P}=V\Rightarrow\Delta T=0$, то есть для идеальных газов эффект Джоуля--Томсона не имеет места. Повышение или понижение температуры реального газа при протекании через пробку при малых $\Delta T$ и $\Delta P$ $\left(\text{для замены их отношения на } \Chpr{T}{P}{I}\right)$ называется \textbf{дифференциальным эффектом Джоуля--Томсона}.\\
Так как течение происходит от большего давления к меньшему, то $\Delta P<0$, значит если $\frac{\Delta T}{\Delta P}>0$, то эффекто Джоуля--Томсона \textbf{положительный}, а если соответственно $\frac{\Delta T}{\Delta P}<0$ --- \textbf{отрицательный}.

\begin{wrapfigure}{R}{4cm}
	\includegraphics[width=40mm]{ris22_2.png}
\end{wrapfigure}
Вычисляя $\chpr{V}{T}{P}=\dfrac{-\Chpr{P}{T}{V}}{\Chpr{P}{V}{T}}$ из уравнения Ван-дер-Ваальса получаем:
$$\dfrac{\Delta T}{\Delta P}=-\dfrac{T\Chpr{P}{T}{V}+V\Chpr{P}{V}{T}}{C_P\Chpr{P}{V}{T}}=\dfrac{\frac{bRT}{(V-b)^2}-\frac{2a}{V^2}}{C_P\Chpr{P}{V}{T}}$$
\paragraph{Температура инверсии.} Пояснение к графику: $T_\infty\equiv T_i,\ H\equiv I$. В случае разреженного газа можно отбросить малые поправки $a$ и $b$ высших порядков $\Rightarrow\dfrac{\Delta T}{\Delta P}=\dfrac{\frac{2a}{RT}-b}{C_P}$. При $T_i=\dfrac{2a}{RB}=\dfrac{27}{4}T_\text{кр.}$ --- температуре инверсии дифференциального эффекта Джоуля--Томсона  $\Delta T=0$. Газ ниже этой температуры охлаждается, выше --- нагревается в процессе Джоуля--Томсона. \\
\paragraph{Интегральный эффект Джоуля--Томсона.} Пусть теперь по разные стороны от перегородки $\Delta P$ велика, тогда велика и $\Delta T$. Тогда мы имеем дело с \textbf{интегральным законом Джоуля--Томсона.}
\begin{equation}
T_2-T_1=\int_{P_1}^{P_2}\chpr{T}{P}{I}dP=\int_{P_1}^{P_2}\dfrac{T\Chpr{V}{T}{P}-V}{C_P}dP
\label{ur22}
\end{equation}
Если во всем диапазоне давлений дифференциального эффекта Джоуля--Томсона положительный, то и интегральный будет положительным. Максимальное охлаждение из начального состояния $T_1,\ P_1$: 
$$\dfrac{d}{dP_1}(T_2-T_1)=\dfrac{dT_2}{dP_1}=-\chpr{T}{P}{I}=0\text{ --- условие максимума записанное при }P=P_1,\ T=T_1$$
Но $\chpr{T}{P}{I}=0$ --- уравнение кривой инверсии дифференциального эффекта Джоуля--Томсона, следовательно, для максимального охлаждения  нужна точка на кривой инверсии дифференциального эффекта Джоуля--Томсона.\\
Формула (\ref{ur22}) интегрируется до конца в случае, когда в начальном состоянии газ под высоким давлением, а после прохода через вентиль его можно рассматривать как идеальный.
$$I_1=I_2\Leftrightarrow\int_{T_0}^{T_1}C_vTdT-\dfrac{a}{V_1}+P_1V_1=\int_{T_0}^{T_2}C_VTdT-\dfrac{a}{V_2}+P_2V_2\ \underset{\tfrac{a}{V_2}\rightarrow0}{\longrightarrow}\int_{T_2}^{T_1}C_VTdT-\dfrac{a}{V_1}+P_1V_1=RT_2$$
$$\overline{C_V}(T_1-T_2)-\dfrac{2a}{V_1}+\dfrac{RT_1V_1}{V_1-b}=RT_2,$$
где $\overline{C_V}$ --- средняя теплоемкость при $V=const$ в диапазоне $T_1\leftrightarrow T_2$. В итоге получим:


\begin{wrapfigure}{R}{52mm}
	\includegraphics[width=50mm]{ris22_3.png}
	\caption{штрих. линия - кривая инверсии диф. эффекта}
\end{wrapfigure}
$$T_2-T_1=\dfrac{1}{R+\overline{C_V}}\left(\dfrac{RbT_1}{V_1-b}-\dfrac{2a}{V_1}\right)$$
Так как знаменатель положительный, то знак эффекта определяется знаком числителя.
\begin{enumerate}
	\item $T_1<\dfrac{2a}{Rb}\dfrac{V_1-b}{V_1}$ --- эффект положительный (охлаждение)
	\item $T_1>\dfrac{2a}{Rb}\dfrac{V_1-b}{V_1}$ --- эффект отрицательный (нагревание)
	\item $T_1=\dfrac{2a}{Rb}\dfrac{V_1-b}{V_1}$ --- температура инверсии интегрального эффекта Джоуля--Томсона
\end{enumerate}
$\tau=27/4\dfrac{\varphi-1/3}{\varphi}$ --- в приведенном виде, $\pi=27-16/27\tau^2\Leftrightarrow\tau=3/4\sqrt{81-3\pi}$

\end{document}
