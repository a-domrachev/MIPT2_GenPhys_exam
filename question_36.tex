\section{\normalsize Классическая теория теплоемкостей. Закон равномерного распределения кинетической энергии теплового движения по степеням свободы. Теплоемкость кристаллов (закон Дюлонга--Пти).}
\paragraph{Теорема о равнораспределении энергии по степеням свободы.} Если макроскопическая система подчиняется законам классической механики, то на каждое слагаемое в энергии, квадратично зависящее от координат приходится энергия $kT/2$ и теплоемкость $k/2$.
\begin{enumerate}
\item Средняя поступательная энергия молекулы. 
Кинетическая энергия поступательного движения молекулы: $\overline{E_\text{пост.}}=3/2kT$, а соответствующая молярная теплоемкость $C=\frac{d\overline{E_\text{пост.}}}{dT}=3/2k\Rightarrow$ в расчете на одну из 3-х степеней свободы:
\[
\overline{E}=kT/2,\ C=k/2\text{ (предполагается } V = const)
\]
\item Средняя вращательная энергия молекулы. Рассмотрим молекулу как твердое тело с главными моментами инерции $I_1,\,I_2,\,I_3\Rightarrow$
\[
 E_\text{вр.}=I_1\omega_1^2/2+I_2\omega_2^2/2+I_3\omega_3^2/2
\]
согласно распределению Гиббса $dW=A\exp(-\frac{E_\text{вр.}}{kT})d\omega_1d\omega_2d\omega_3\Rightarrow$
\[
\overline{\omega_i^2}=\frac{kT}{I_i},\ i=\overline{1,3}\Rightarrow
\]
на каждую вращательную степень свободы - по $kT/2$.
\item Колебательная энергия двухатомной молекулы. Можем рассматривать как осциллятор с $\omega=\sqrt{\frac{\varkappa}{\mu}}$, $\varkappa$ --- жесткость связи, $\mu$ --- приведенная масса атомов.\\
Если между атомами $r$, а $r_0$ --- в равновесии, то $x=r-r_0\Rightarrow$
\[
E=\mu v_\text{отн.}^2/2+\mu\omega^2x^2/2
\]
первое слагаемое --- кинетическая энергия относительного движения атомов,\\
второе --- потенциальная энергия их взаимодействия между собой.
\end{enumerate}
\[
\overline{E_\text{кол.}}=\frac{\int E\exp(-E/kT)d\Gamma}{\int e\exp(-E/kT)d\Gamma}=\frac{\int E\exp(-\beta E)d\Gamma}{\int \exp(-\beta E)d\Gamma}=-\frac{\partial \ln z}{\partial \beta}
\]
$ d\Gamma=dv_\text{отн.}=dv_\text{отн.} dx$ --- элемент фазового пространства.
$\beta = \frac{1}{kT},\ z=\int \exp(-\beta E)dv_\text{отн.}dv$ --- статистическая сумма.
\[
z=\int \exp \left(-\beta \mu v_\text{отн.}^2/2 - \beta \omega^2x^2/2 \right)dv_\text{отн.}dx=z_\text{кин.}z_\text{пот.}
\]
\begin{equation*}
	\begin{cases}
	z_\text{кин.}=\infint\exp(-\beta \mu v_text{отн.}^2/2)dv_\text{отн.}=\sqrt{\frac{2\pi}{\beta \mu}}\\
	z_\text{пот.}=\infint\exp(-\beta \mu \omega^2x^2/2)dx=\sqrt{\frac{2\pi}{\beta \mu \omega^2}}
	\end{cases}\Rightarrow
\end{equation*}
\[
\Rightarrow \overline{E_\text{кол.}}=-\frac{\partial \ln z}{\partial \beta}=-\frac{\partial \ln z_\text{кин.}}{\partial \beta}-\frac{\partial \ln z_\text{пот.}}{\partial \beta}=kT
\]
таким образом, на 1 колебательную степень свободы приходится энергия $kT$: по $kT/2$ на кинетическую и потенциальную энергии.
Таким образом, для двухатомной молекулы:
\[
\overline{E}=\overline{E_\text{пост.}}+\overline{E_\text{вращ.}}+\overline{E_\text{кол.}}=7/2 kT;\ C_V=7/2k
\]
\paragraph{Закон Дюлонга-Пти.} Твердое тело (кристалл) представляет собой совокупность атомов, находящихся в окрестности своего положения равновесия. Атомы могут колебаться в трех направлениях, так что каждый атом в среднем $3kT\Rightarrow$ на 1 моль: $E=3N_AkT$ и $C_V=3N_AK=3R$ --- это экспериментально установленное соотношение и есть \textbf{Закон Дюлонга--Пти}.