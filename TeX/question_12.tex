\section{\normalsize Свободная энергия Гельмгольца, термодинамический потенциал Гиббса. Уравнения Гиббса--Гельмгольца.}
\paragraph{Свободная энергия Гельмгольца.} $$\Psi=\Psi(T,V)=U-TS\then d\Psi=-PdV-SdT\then -S=\left(\dfrac{\partial \Psi}{\partial T}\right)_V,\;-P=\left(\dfrac{\partial\Psi}{\partial V}\right)_T\then$$ $$\then\dfrac{\partial^2 \Psi}{\partial V\partial T}=\dfrac{\partial^2 \Psi}{\partial T\partial V}\Leftrightarrow\dfrac{\partial}{\partial V}\left(\dfrac{\partial \Psi}{\partial T}\right)_V=\dfrac{\partial}{\partial T}\left(\dfrac{\partial \Psi}{\partial V}\right)_T\Leftrightarrow\underline{\left(\dfrac{\partial S}{\partial V}\right)_T=\left(\dfrac{\partial P}{\partial T}\right)_V}$$
Вот такой метод получения данных соотношений через двойное дифференцирование и называется \textbf{методом Максвелла}.
\paragraph{Термодинамический потенциал Гиббса.} $$\Phi=\Phi(T,P)=U-TS+PV\then d\Phi=VdP-SdT\then V =\chpr{\Phi}{P}{T},\;-S=\chpr{\Phi}{T}{P}\then$$
$$\then\vtchpr{T}{\Phi}{P}{T}=\vtchpr{P}{\Phi}{T}{P}\Leftrightarrow\underline{\chpr{V}{T}{P}=-\chpr{S}{P}{T}}$$
\paragraph{Уравнения Гиббса-Гельмгольца.} Из свободной энергии Гельмгольца $U = \Psi + TS$ подставим $S$ полученное при частном дифференцировании, тогда $U = \Psi -T\chpr{\Psi}{T}{V}$. Проведем аналогичные действия с термодинамическим потенциалом Гиббса: \\$I = \Phi - T\chpr{\Phi}{T}{P}$. Данные соотношения называются \textbf{Уравнениями Гиббса-Гельмгольца}
