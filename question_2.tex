\section{\normalsize Работа, внутренняя энергия, теплота. Первое начало термодинамики.}

\paragraph{Работа.} Бесконечно малая \textbf{элементарная работа} $\delta A$, совершаемая газом при бесконечно малом квазистатическом расширении, в котором его объем увеличивается на $dV$ рассматриваемая в модели газа под поршнем. $F = PS$ ($P$ - const, т.к. перемещение малое) $\then$ при перемещении поршня на $dx:\,\delta A = F dx= PSdx=PdV $. Следует еще заметить , что в квазистатических процессах $\delta A = - \delta A_\text{внешн.}$.\\
\textbf{Работа конечного процесса}: $A = \int_{1 \rightarrow 2} \delta A $, она не является функцией состояния, т.к. зависит от пути перехода от 1 к 2.\\
\textbf{Адиабатическая оболочка} характерна тем, что при любых изменениях температуры окружающих тел состояние системы внутри оболочки неизменно. Значение всех прочих внешних параметров неизменны, например не совершается механическая работа. Изменить состояние системы можно путем механической работы.
\paragraph{Внутренняя энергия.} \textbf{Внутренней энергией $U$ системы} называется функция состояния, приращение которой во всяком процессе, совершенном системой в адиабатической оболочке, равно работе внешних сил над системой при переходе ее из начального равновесного состояния в конечное, также равновесное, т.е. $U_2 - U_1 = A_\text{внешн.} \rightarrow U$ --- функция состояния.
\paragraph{Теплота.} Пусть система заключена в жесткую теплопроводную оболочку $\then$ имеем чисто тепловой контакт системы  с внешней средой без совершения макроскопической работы, происходит \textbf{теплообмен}, сопровождающийся обменом внутренними энергиями соприкасающихся тел, т.е. \textbf{количество теплоты} --- приращение внутренней энергии в процессе чистого теплообмена. $Q = U_2 - U _1$ --- полученное тепло (не функция состояния!)
\paragraph{Первое начало термодинамики.} Теплота Q, полученная системой, идет на приращение ее внутренней энергии $\Delta U = U_2 - U_1$, и совершение системой работы
$$ \int_{1 \rightarrow 2}\delta Q = \int_{1 \rightarrow 2}dU + \int_{1 \rightarrow 2}\delta A \then Q = \Delta U + A$$
Если процесс круговой, то $U_1 = U_2$ и $Q=0$, то $A = 0 \then$ невозможен процесс, единственным результатом которой является производство работы без каких-либо изменений в других телах.
