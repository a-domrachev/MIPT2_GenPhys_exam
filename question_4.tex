\section{\normalsize Работа идеального газа в равновесных изотермическом и изобарическом процессах. Внутренняя энергия идеального газа.}
\paragraph{Работа идеального газа в равновесных изотермическом и изобарическом процессах.} 
Работа $\nu = 1$ моль идеального \textbf{изотермическом} расширении. $PV = RT =\\ = const \then A = \int_{V_1}^{V_2}PdV=RT\int_{V_1}^{V_2}\dfrac{dV}{V}=RT \ln(\dfrac{V_2}{V_1})\then A_T=RT \ln(\dfrac{V_2}{V_1})$ \\
Работа $\nu = 1$ моль идеального газа при \textbf{изобарном} расширении. $$\delta A = PdS\cdot dn = P dV_\text{эл.} \then \delta A_P = \int_{V_\text{слоя}}PdV_\text{эл.} = P \int_{V_\text{слоя}} dV_\text{эл.} = PdV \then A_P=P\Delta V$$
\paragraph{Внутренняя энергия идеального газа.} Опыт Джоуля: идеальный газ с температурой $T_1$, давлением $P_1$ находится в части адиабатической оболочки объемом $V_1$, вторая часть оболочки откачана до вакуума. Перегородку между частями убирают, в следствии расширения газа его температура не изменилась, а изменилось лишь давление и объем. Рассмотрим это эмпирическое наблюдение с точки зрения теории. По первому началу термодинамики  $Q = \Delta U + A$, причему $Q=0$ , т.к. газ находится в адиабатической оболочке и $A=0$, т.к. газ расширялся в вакуум $\then \Delta U = 0 \then U_2(T,V_2)=U_1(T,V_1) \then$ функция $U$ зависит лишь от второго параметра и $\left(\dfrac{\partial U}{\partial V}\right)_T = 0$ для идеального газа. Внутренняя энергия идеального газа зависит лишь от температуры, поскольку она определяется лишь кинетической энергией молекул.\\
Для одноатомного газа: $U = N\left<\dfrac{mv^2}{2}\right>+ U_0$, пусть для удобства $U_0=0 \then U= \\= N \dfrac{3}{2}kT = \dfrac{3}{2}\nu N_A kT = \dfrac{3}{2}\nu RT$ \\
В общем случае: $U = \dfrac{i}{2}\nu RT$, где $i$ - степени свободы газа.
