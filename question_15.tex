\section{\normalsize Теплофизические свойства твердых тел. Адиабатическое растяжение упругого стержня.}
\paragraph{Теплофизические свойства твердых тел.} \textbf{Теплофизические свойства материала} -- свойства, характеризующие поведение этого материала при изменении температуры, как-либо: теплоемкость, теплопроводность, коэффициенты теплового расширения, температура плавления.
\paragraph{Адиабатическое растяжение упругого стержня.} \textbf{Уравнение состояния} --- $f~=~f(l,T)$. При $f=0$ $l(T,0)=l_0(1+\alpha(T-T_0))$, где $l_0=l(T_0,0)$, $\alpha$ --- коэффициент линейного температурного расширения. При $T=const$ согласно закону Гука: $$\dfrac{\Delta l}{l}=\dfrac{f}{ES_\perp}=\dfrac{l(T,f)-l(T,0)}{l(T,0)}\then f=ES_\perp\left(\dfrac{l}{l_0(1+\alpha[T-T_0])}-1\right)$$
В большинстве случаев тепловая деформация мала, $\alpha|T-T_0|\ll1\then$
$$\then f=ES_\perp\left(\dfrac{l}{l_0}(1-\alpha[T-T_0])-1\right)$$
Предположен, что стержень окружен адиабатической оболочкой. При квазистатической деформации $dS=\chpr{S}{T}{l}dT+\chpr{S}{l}{T}dl=0\Rightarrow dT=-\dfrac{\Chpr{S}{l}{T}}{\Chpr{S}{T}{l}}dl;\;\chpr{S}{T}{l}=\left(\dfrac{\delta Q}{\partial T}\right)_l~\dfrac{1}{T}~=~\dfrac{C_l}{T}$ \\
$$PdV=-\sigma d(s_\perp l)=-\sigma S_\perp dl=-fdl\Rightarrow dU=TdS+fdl$$
$$\Psi=U-TS,\,d\Psi=-SdT+fdl\Rightarrow\text{ метод Максвелла --- } \chpr{S}{l}{T}=\chpr{f}{T}{l}$$
Откуда 
$$dT=\dfrac{T}{C_l}\chpr{f}{T}{l}dl$$
Используя $f:\,\Delta T=\int_{l_0}^{l}\dfrac{T}{C_l}\chpr{f}{T}{l}dl=-\dfrac{ES_\perp \alpha}{2C_ll_0}T(l^2-l_0^2)\simeq\underline{-\dfrac{ES_\perp\alpha}{C_l}T(l-l_0)}$\\
При адиабатическом растяжении ($l>l_0$) температура стержня понижается из-за совершения работы против внутренних сил притяжения молекул. Для идеального стержня ($E=const$):
$$U_\text{деф.}=V\dfrac{E\varepsilon^2}{2}=\Psi_\text{деф.}$$
Внутренняя энергия деформации совпадает со свободной энергией и явно не зависит от температуры.
